

\begin{longtable}{lp{10cm}}


Time        & Plans  \\                                                          
\hline
01.19-01.22 & Meet the members of the group and improve the information needed on Git hub; Discuss the main objectives of the project together; Divide the members of the group to define their tasks  \\
%分工
01.22-01.31 & Make a preliminary framework and make a demo of this synchronizer. Write the initial report and prepare the materials needed for presentation         \\       
%初步成果讨论                                                                                                             
01.31-02.09 & Build a rough framework of the mechanism which may include selected design patterns, base class wrappers and extended category layers; Implement specific functions such as launching log-ins and real-time editing                                    \\
02.09-02.25 & Test mobile phone clients on the simulator and then debug; Start writing reports                                                                                                                                          \\
02.25-03.19 & Modify the final report and prepare for the final presentation                                                                                                                  \\
03.19-03.27 & Submit the final report     \\                                              
\hline

\end{longtable}

   % 1月22日(刚分完小组),认识组内成员,一起讨论项目的主要目标,回去搜集资料,提出自己认为需要组内成员掌握的知识和技,完善github上所需的各项信息
   % 1月29号 为组内成员进行分工,每个人平均公平地分得自己的任务
   %2月5号,有一个初步的框架并做出这个同步器的demo。撰写中期report并准备presentation所需的材料
   
   %2月19号搭建好大概的框架(选定设计模式,基类封装,扩展类别层)
   %3月5号实现具体功能,例如启动登录还有实时编辑等
   %3月19号 测试 ios端就在模拟器上实验 然后debug 开始撰写报告 各种查漏补缺
   %3月28日提交finalreport